% !TeX root = ../main.tex
\chapter{Conclusion}

This document explained the basic principles behind passive radar. First a brief summary of technological history was presented. It can be seen, that while interest in passive radar technology was intermittent, its merits clearly warrant continued investigation.
In Chapter~\ref{chp:theory_of_operation} first idiosyncrasies of bistatic radar were discussed. It must be noted, that bistatic geometry produces several challenges, that need to be overcome, in order to produce an effective radar. Next the process of signal correlation was introduced and a practical method of computation was presented. Lastly, further processing steps were outlined, concluding the signal processing chain for passive radar systems.

The following will provide a brief summary of passive radars weaknesses and strengths, as well as outline possible subjects of future work.

\section{Limitations and Benefits}

It should come as no surprise that passive radar heavily relies on the existence of stable illuminators of opportunity. Without them, or in case of their interruption, passive radar is blind.

However, its biggest weakness is also its greatest strength. Especially military interest groups are keen on passive radars lack of radio emissions, and thus low probability of detection. Its covert nature allows air surveillance operations, without fear of detection, jamming or destruction through means of electronic warfare or radiation seeking missiles. It also offers greater probability of detecting stealth aircraft, due to its bistatic geometry and operation in lower frequency bands~\cite[p.~2]{Malanowski2019}.

\section{Future Work}

Passive radar currently experiences a phase of relatively high interest, driven mainly by military use-cases that rest upon passive radar's lightweight, high mobility and low probability of detection. However, also the technology may have reached a level of maturity where civil applications can start to emerge. Further work will likely revolve around identification of additional illuminators of opportunity, and their introduction into practical use-cases.

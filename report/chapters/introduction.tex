% !TeX root = ../../report.tex
\chapter{Introduction}\label{chp:introduction}

Today's electromagnetic wave spectrum is sectioned into bands. While certain bands are inherently limited in their usefulness, due to atmospheric absorption, meteorologic, geologic or cosmic interference, most remaining bands are allocated to certain use-cases by regulatory institutions. The list of use-cases is immense and may range from radio astronomy, amateur radio, to maritime or aeronautical navigation, \emph{including radar}. However, possibly the largest holder of radio spectrum is \emph{telecommunications}. This reflects our societies growing demand for ever faster communications- and data-transfer-services. Sometimes this need conflicts or supersedes other use-cases. As with the recent 5G expansions in the L- and C-band, causing concerns for neighboring GPS~\cite{GPS.gov2020} and radar altimeter~\cite{RTCA2020} interest groups about their uninterrupted operability.

In the future, active radar system designers will likely have to fight for further spectrum-shares and defend against interference and displacements by a growing telecommunication interest. Whereas, passive radar systems, by nature, do not have to fight for their own spectrum. Contrary to active radar, passive radar can in fact piggyback off large telecom radio networks, and in a sense \textquote[{\cite[p.~1]{Malanowski2019}}]{recycle} ambient radiations from radio broadcasts and other types of emissions. Another benefit lies in passive radars inherent covertness, as such a system does not rely on emissions of its own. This on the other hand, makes passive radar heavily reliant on suitable illuminators in the vicinity.

\section{Definition}

As the term \emph{passive} implies, passive radar systems (also known as passive covert radar, passive coherent location or passive bistatic radar) do not emit their own radio energy in order to illuminate a target. Instead, they rely on other \emph{ambient} illuminators, which depending on type, may vary from conventional air surveillance radar, FM-radio, DAB, DVB-T, to Wi-Fi and Cellular~\cite[pp.~48--77]{Malanowski2019}\ \cite{Kuschel2013}. Due to the considerable spatial separation between sensor and illuminator passive radar is considered bistatic in nature. Figure~\ref{fig:topology} depicts the basic topology of a bistatic system, as employed by most passive radar systems. An illuminator sends out electromagnetic waves in the radio to microwave spectrum. These waves hit a target of interest, which will itself radiate a small portion of the received power back into the environment. The echo, as well as direct signal coming from the illuminator, arrive at the receiver with some delay and (in case the target is moving) Doppler frequency shift. Using intricate signal processing, a range and velocity is determined, details of which will be discussed in Chapter~\ref{chp:theory_of_operation}.

\begin{figure}
    \centering
    \begin{tikzpicture}
        \newcommand\drawTopology[4]{
    \coordinate (rx1_coord) at (-2,0);
    \coordinate (tx1_coord) at (2,0);
    \coordinate (target_coord) at (1.25,3);

    \node at (tx1_coord) [text width=0.5cm,text height=1cm] (tx) {};
    \node at (tx1_coord) [antenna,scale=0.5,below=-0.5cm] {};
    \path let \p1=(tx.north west), \p2=(tx.south east), \p3=(tx) in [label={below:Receiver}] ({\x3-0.4cm},\y1) rectangle ({\x3+0.4cm},\y2) node [below] at (\x3,\y2) {Illuminator};

    \node at (rx1_coord) [text width=0.5cm,text height=0.75cm] (rx) {};
    \node at (rx1_coord) [antenna,scale=0.5,below=-0.5cm] {};
    \path let \p1=(tx.north west), \p2=(tx.south east), \p3=(rx) in [label={below:Receiver}] ({\x3-0.4cm},\y1) rectangle ({\x3+0.4cm},\y2) node [below] at (\x3,\y2) {Receiver};

    \node at (target_coord) [label={above:Target}] (target) {\Huge\faPlane};

    \draw [visible on={#1},decorate,decoration={expanding waves,angle=35,segment length=4pt},blue!60] (tx) -- (target);
    \draw [visible on={#2},decorate,decoration={expanding waves,angle=25,segment length=4pt},red!60] (target) -- node [#4] {\small\contourlength{2pt}\contour{#3}{Echo}} (rx);
    \draw [visible on={#1},decorate,decoration={expanding waves,angle=15,segment length=4pt},blue!60] (tx) -- node [#4] {\small\contourlength{2pt}\contour{#3}{Reference}} (rx);
}

    \end{tikzpicture}
    \caption{Basic topology of a bistatic system.}\label{fig:topology}
\end{figure}

\section{History}

The first known system classifiable as passive radar dates back to Sir Robert Watson-Watt's \emph{Daventry Experiment} from 1935. It involved a \SI{10}{\kilo\watt} BBC radio transmitter, used as illuminator to detect a designated Heyford bomber~\cite[pp.~5--6]{Malanowski2019}~\cite{Kuschel2013}. Later, during the \nth{2} World War, Britain constructed the famous Chain Home radar network, which was an active radar system. A lesser known fact is that in 1942 the Germans built the Klein Heidelberg system. Following the passive radar approach, they detected incoming British aircraft, using their enemies Chain Home radar as illuminator. While the system demonstrated typical benefits associated with passive radar, i.e.\ resilience to certain forms of jamming, by means of chaff, its inception occurred too late as to have any deciding effect on the war~\cite{Griffiths2010}.

After WW2 focus shifted on active radar. Only around the 1980s interest was resurrected in the form of so called pulse chasing passive radar systems. These rely on emissions by  conventional active air surveillance radars and try to \emph{chase} along their pulses with an electronically steered antenna beam. Finally, by the 1990s work began on covert FM-radio based passive radar. This effort later expanded to support DAB and DVB-T emitters~\cite{Kuschel2013}. Their relatively high transmit power, in the order of few (DAB), several tens (DVB-T) to hundred \si{\kilo\watt} (FM)~\cite[p.~48, 57, 64]{Malanowski2019}, makes them comparable in power to active radar, when averaging pulse energy over pulse repetition interval.

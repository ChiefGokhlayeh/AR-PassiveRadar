% !TeX root = ../beamer.tex
\section{Zusammenfassung}

\begin{frame}
    \frametitle{Zusammenfassung}

    \begin{columns}
        \begin{column}{0.4\textwidth}
            \begin{itemize}
                \item Begriffsdefinition Passivradar
                \item Monostatische und bistatische Geometrie
                \item Bistatische Entfernung und Geschwindigkeit
                \item Methoden zur Umwandlung zwischen bistatischer und monostatischer Geometrie
            \end{itemize}
        \end{column}
        \begin{column}{0.6\textwidth}
            \centering
            \begin{adjustbox}{max width=\linewidth,max height=0.3\textheight}
                \begin{tikzpicture}
                    \newcommand\drawTopology[4]{
    \coordinate (rx1_coord) at (-2,0);
    \coordinate (tx1_coord) at (2,0);
    \coordinate (target_coord) at (1.25,3);

    \node at (tx1_coord) [text width=0.5cm,text height=1cm] (tx) {};
    \node at (tx1_coord) [antenna,scale=0.5,below=-0.5cm] {};
    \path let \p1=(tx.north west), \p2=(tx.south east), \p3=(tx) in [label={below:Receiver}] ({\x3-0.4cm},\y1) rectangle ({\x3+0.4cm},\y2) node [below] at (\x3,\y2) {Illuminator};

    \node at (rx1_coord) [text width=0.5cm,text height=0.75cm] (rx) {};
    \node at (rx1_coord) [antenna,scale=0.5,below=-0.5cm] {};
    \path let \p1=(tx.north west), \p2=(tx.south east), \p3=(rx) in [label={below:Receiver}] ({\x3-0.4cm},\y1) rectangle ({\x3+0.4cm},\y2) node [below] at (\x3,\y2) {Receiver};

    \node at (target_coord) [label={above:Target}] (target) {\Huge\faPlane};

    \draw [visible on={#1},decorate,decoration={expanding waves,angle=35,segment length=4pt},blue!60] (tx) -- (target);
    \draw [visible on={#2},decorate,decoration={expanding waves,angle=25,segment length=4pt},red!60] (target) -- node [#4] {\small\contourlength{2pt}\contour{#3}{Echo}} (rx);
    \draw [visible on={#1},decorate,decoration={expanding waves,angle=15,segment length=4pt},blue!60] (tx) -- node [#4] {\small\contourlength{2pt}\contour{#3}{Reference}} (rx);
}

                    \drawTopology{<1->}{<1->}{bg}{fg}
                \end{tikzpicture}
            \end{adjustbox}
            \begin{adjustbox}{max width=\linewidth,max height=0.3\textheight}
                \begin{tikzpicture}
                    % !TeX root = ../main.tex
\coordinate (rx1_coord) at (-2,0);
\coordinate (tx1_coord) at (2,0);
\coordinate (target_coord) at (1,1.25);

\node at (tx1_coord) [draw,fill=white] (tx1) {Tx};

\node at (rx1_coord) [draw,fill=white] (rx) {Rx};

\begin{scope}
    \clip
    let
    \p1=(target_coord),
    in
    (\x1 - 0.75cm,\y1 + 0.75cm) rectangle +(1.5,-1.5);
    \draw [color=gray]
    (2.42539052968,0) arc [start angle=0,end angle=360,x radius=2.42539052968,y radius=1.37204927807];
\end{scope}

\node at (target_coord) [label={[fill=white]above:Target}] (target) {\faPlane};

\draw [->,color=red] (tx1) -- node [black,midway,right,align=center] {$R_1$} (target);
\draw [->,color=red] (target) -- node [black,midway,left=4pt,align=center] {$R_2$} (rx);
\draw [->,color=blue] (tx1) -- node [black,midway,below,align=center] {$R_{\text{b}}$} (rx);

\pic [draw,angle radius=1cm,"$\beta$"] {angle=rx1_coord--target_coord--tx1_coord};

                    \drawBistaticGeometry{bg}
                \end{tikzpicture}
            \end{adjustbox}
            \begin{adjustbox}{max width=\linewidth,max height=0.25\textheight}
                \begin{tikzpicture}
                    
\def\F{5}
\coordinate (l_foci_coord) at (-\F,0);
\coordinate (r_foci_coord) at (\F,0);
\coordinate (point_on_ellipse_coord) at (6,0);

\draw (0,0)
let
\p1=($(point_on_ellipse_coord)-(r_foci_coord)$),
\p2=($(point_on_ellipse_coord)-(l_foci_coord)$),
\n1={scalar((veclen(\x1,\y1) + veclen(\x2,\y2))*1pt/1cm)},
\n2={\n1/2},
\n3={sqrt(pow(\n1/2, 2) - pow(\F, 2))}
in
circle [x radius=\n2,y radius=\n3,draw=gray];

\draw [dotted] (l_foci_coord) node [cross out,draw,solid] {} -- (0,0) node {\contour{bg}{$R_{\text{b}}$}} -- (r_foci_coord) node [cross out,draw,solid] {};

\foreach \u in {-5.5,-5,...,5.5} {
        \draw [->,blue] let
        \p1=($(point_on_ellipse_coord)-(r_foci_coord)$),
        \p2=($(point_on_ellipse_coord)-(l_foci_coord)$),
        \n1={scalar((veclen(\x1,\y1) + veclen(\x2,\y2))*1pt/1cm)},
        \p3=(\u,{sqrt( (pow(2 * \F / \n1, 2) - 1) * pow(\u, 2) + pow(\n1 / 2, 2) - pow(\F, 2) )}),
        \p4=($(\p3)!1cm!(l_foci_coord)$),
        \p5=($(\p3)!1cm!(r_foci_coord)$),
        \p6=($(\p4) + (\p5) - (\p3)$),
        in
        (\p3) -- ($(\p6)!1.5!(\p3)$);
    }

\foreach \u in {-6,6} {
        \draw [->,blue] let
        \p1=(\u,0),
        \p2=($(\p1)!1cm!(l_foci_coord)$),
        \p3=($(\p1)!1cm!(r_foci_coord)$),
        \p4=($(\p2) + (\p3) - (\p1)$),
        in
        (\p1) -- ($(\p4)!1.5!(\p1)$);
    }

\foreach \u in {-5.5,-5,...,5.5} {
        \draw [->,blue] let
        \p1=($(point_on_ellipse_coord)-(r_foci_coord)$),
        \p2=($(point_on_ellipse_coord)-(l_foci_coord)$),
        \n1={scalar((veclen(\x1,\y1) + veclen(\x2,\y2))*1pt/1cm)},
        \p3=(\u,{-sqrt( (pow(2 * \F / \n1, 2) - 1) * pow(\u, 2) + pow(\n1 / 2, 2) - pow(\F, 2) )}),
        \p4=($(\p3)!1cm!(l_foci_coord)$),
        \p5=($(\p3)!1cm!(r_foci_coord)$),
        \p6=($(\p4) + (\p5) - (\p3)$),
        in
        (\p3) -- ($(\p6)!1.5!(\p3)$);
    }

                \end{tikzpicture}
            \end{adjustbox}
        \end{column}
    \end{columns}

    \vspace{\baselineskip}

    \centering
    \huge Danke für Ihre Aufmerksamkeit! \normalsize
\end{frame}

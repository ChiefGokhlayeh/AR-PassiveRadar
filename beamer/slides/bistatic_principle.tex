% !TeX root = ../../beamer.tex
\section{Bistatisches Prinzip}

\begin{frame}
    \frametitle{Monostatische Geometrie}

    \begin{figure}
        \centering
        \begin{tikzpicture}
            \coordinate (horizon) at (3,0);

            \begin{scope}
                \clip (0,0) rectangle ({sqrt(2*pow(3,2))},{sqrt(2*pow(3,2))});
                \draw (0,0) circle [radius={sqrt(2*pow(3,2))}];
            \end{scope}

            \node [label={below:Radar}] (radar) at (0,0) {\Huge\faSatelliteDish};

            \node [label={above:Target}] (target) at (3,3) {\Huge\faPlane};

            \draw [<->,red] (radar) -- (target);
            \draw [dotted] (radar) -- (horizon);

            \pic [draw,angle radius=1.5cm,angle eccentricity=0.8,"$\phi$"] {angle=horizon--radar--target};
        \end{tikzpicture}
        \caption{Monostatische Geometrie bei konventionellem Radar.}
    \end{figure}
\end{frame}

\begin{frame}
    \frametitle{Bistatische Geometrie}

    \begin{figure}
        \centering
        \resizebox{0.7\textwidth}{!}{
            \begin{tikzpicture}
                % !TeX root = ../main.tex
\coordinate (rx1_coord) at (-2,0);
\coordinate (tx1_coord) at (2,0);
\coordinate (target_coord) at (1,1.25);

\node at (tx1_coord) [draw,fill=white] (tx1) {Tx};

\node at (rx1_coord) [draw,fill=white] (rx) {Rx};

\begin{scope}
    \clip
    let
    \p1=(target_coord),
    in
    (\x1 - 0.75cm,\y1 + 0.75cm) rectangle +(1.5,-1.5);
    \draw [color=gray]
    (2.42539052968,0) arc [start angle=0,end angle=360,x radius=2.42539052968,y radius=1.37204927807];
\end{scope}

\node at (target_coord) [label={[fill=white]above:Target}] (target) {\faPlane};

\draw [->,color=red] (tx1) -- node [black,midway,right,align=center] {$R_1$} (target);
\draw [->,color=red] (target) -- node [black,midway,left=4pt,align=center] {$R_2$} (rx);
\draw [->,color=blue] (tx1) -- node [black,midway,below,align=center] {$R_{\text{b}}$} (rx);

\pic [draw,angle radius=1cm,"$\beta$"] {angle=rx1_coord--target_coord--tx1_coord};

                \drawBistaticGeometry{bg}
            \end{tikzpicture}
        }
        \caption{Baseline \(R_{\text{b}}\), Entfernung: Sender/Ziel \(R_{1}\), Entfernung: Ziel/Empfänger \(R_{2}\), Bistatischer Winkel \(\beta\), Bewegungsrichtung Ziel \(\delta\), Geschwindigkeitsvektor \(v\).}
    \end{figure}
\end{frame}
